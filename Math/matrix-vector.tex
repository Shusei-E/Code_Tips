\documentclass[a4paper,10.5pt]{jsarticle}  %jsarticleでも良いかも
%--脚注の設定
\usepackage{natbib}
\bibpunct[, ]{(}{)}{;}{and}{}{,} %本文での引用の体裁はここで整えられるみたい
\bibliographystyle{apsr2006-2}   %このagsmは編集済みなので注意。ジャーナルが太字にならないようにしてる。ブログ参照。
\usepackage{amsmath}
%--余白の設定   http://pcwide-jp.blogspot.co.uk/2009/07/latex.html  より
\setlength{\topmargin}{20mm}
\addtolength{\topmargin}{-1in}
\setlength{\oddsidemargin}{20mm}
\addtolength{\oddsidemargin}{-1in}
\setlength{\evensidemargin}{15mm}
\addtolength{\evensidemargin}{-1in}
\setlength{\textwidth}{170mm}
\setlength{\textheight}{254mm}
\setlength{\headsep}{0mm}
\setlength{\headheight}{0mm}
\setlength{\topskip}{0mm}
%--図の設定
\usepackage[dvipdfmx]{graphicx} % PDFの利用もOKに
  %\usepackage[dvips]{graphicx}
%--行間の設定
\usepackage{setspace} 
\setstretch{1.13} % ページ全体の行間を設定
%コードの設定
\usepackage{listings}
\usepackage{color}
\definecolor{dkgreen}{rgb}{0,0.6,0}
\definecolor{mygray}{rgb}{0.5,0.5,0.5}
\definecolor{mauve}{rgb}{0.58,0,0.82}

\definecolor{codegreen}{rgb}{0,0.6,0}
\definecolor{codegray}{rgb}{0.5,0.5,0.5}
\definecolor{codepurple}{rgb}{0.58,0,0.82}
\definecolor{backgroundcolour}{rgb}{0.95,0.95,0.92}

\lstset{ %
  language= Python, %ここを含めなければ、変に太字になったりしない
  aboveskip=2.5mm,
  belowskip=4.5mm,
  showstringspaces=false,
  columns=flexible,
  keepspaces=true,
  numbers=left,                    
  numbersep=5pt,    
  basicstyle={\small\ttfamily},
  commentstyle={\small\ttfamily},
  breaklines=true,
  breakatwhitespace=true
  tabsize=3
  backgroundcolor=\color{backgroundcolour},   
  commentstyle=\color{codegreen},
  keywordstyle=\color{magenta},
  numberstyle=\tiny\color{codegray},
  stringstyle=\color{codepurple},
	xleftmargin = 1.1cm,
  framexleftmargin = 1em
}
\usepackage{xcolor}
\usepackage{framed}
\colorlet{shadecolor}{green!8}
%リンクの埋め込みを可能にする  \href{ **URL** }{表示テキスト}
\usepackage[dvipdfmx]{hyperref}
\usepackage{courier}
\usepackage{here}
%日本語環境用の設定を追加
\renewcommand{\refname}{参考文献}
\renewcommand{\abstractname}{要約}
\renewcommand{\contentsname}{Table of Contents}
\usepackage{multirow}
\usepackage{placeins}
%数学用のフォント
\usepackage{amsmath} 
\usepackage{amssymb}
\usepackage{amsfonts} 
%-----------------
\begin{document}

\title{Matrix and Vector}
\author{}
\date{}
\maketitle
%\begin{abstract}
%\end{abstract}

\setcounter{tocdepth}{1}
\tableofcontents

\section{Calculation of the squared Euclidean norm}
From \href{http://math.stackexchange.com/questions/1865476/calculation-of-the-squared-euclidean-norm/1865487}{StackExchange}.
\subsection{Question}
\begin{align*}
&\|\mathbf{x} - \mathbf{\alpha} \|^2 - \|\mathbf{x} - \mathbf{\beta}\|^2\\
&= \|\mathbf{x}\| \|\mathbf{x}\| - 2 \|\mathbf{\mathbf{\alpha}}\| \|\mathbf{x}\| + \|\mathbf{\alpha}\| \|\mathbf{\mathbf{\alpha}}\|  - \|\mathbf{x}\| \|\mathbf{x}\| +2 \|\mathbf{\beta}\| \|\mathbf{x}\| - \|\mathbf{\beta}\| \|\mathbf{\beta}\|\\
&= \mathbf{\alpha}^T \mathbf{\alpha} - \beta^T \beta + 2(\sqrt{\mathbf{\beta}\cdot\mathbf{\beta}}-\sqrt{\mathbf{\alpha}\cdot\mathbf{\alpha}})\|\mathbf{x}\|\\
&= \mathbf{\alpha}^T \mathbf{\alpha} - \beta^T \beta + 2(\sqrt{\mathbf{\beta}\cdot\mathbf{\beta}}-\sqrt{\mathbf{\alpha}\cdot\mathbf{\alpha}})(\sqrt{\mathbf{x}\cdot\mathbf{x}}),
\end{align*}
where I used the fact that $$\|a\|\|a\| = \sqrt{a\cdot a}\sqrt{a\cdot a} = \sqrt{a^T a}\sqrt{a^T a} = a^Ta.$$

However, the article gives $$2(\beta-\alpha)^T \mathbf{x} + \alpha^T \alpha - \beta^T\beta $$

\subsection{Answer}
Your transition from the first line to the second is incorrect.  We should have
\begin{align*}
  &\|x - \alpha\|^2 - \|x - \beta\|^2\\
  &(x - \alpha)^T(x - \alpha) - (x - \beta)^T(x - \beta)  \\
  &=\|x\|^2 + \|\alpha\|^2 - \|x\|^2 - \|\beta\|^2 - x^T\alpha - \alpha^Tx + x^T\beta + \beta^Tx \\
  &=\|\alpha\|^2 - \|\beta\|^2 - 2\alpha^Tx + 2 \beta^Tx \\
  &=\alpha^T\alpha - \beta^T\beta + 2(\beta - \alpha)^Tx
\end{align*}
which is the desired result.

\subsection{Caution}
\subsubsection{}
For any vectors $u, v$, $u^Tv = v^Tu$.

\subsubsection{}
\begin{align*}
  (x-\alpha)^T (x-\alpha) = x^T x - x^T \alpha - \alpha^T x + \alpha^T \alpha 
\end{align*}

\section{General answer of $(\mathbf{x}_n - \boldsymbol{\mu}_k)^\top \boldsymbol{\Lambda}_k(\mathbf{x}_n - \boldsymbol{\mu}_k)$}
From \href{http://math.stackexchange.com/questions/1913182/general-answer-of-mathbfx-n-boldsymbol-mu-k-top-boldsymbol-lambda/1913208#1913208}{StackExchange}.
\subsection{Question}
Can we say this generally? 
$$(\mathbf{x}_n - \boldsymbol{\mu}_k)^\top \boldsymbol{\Lambda}_k(\mathbf{x}_n - \boldsymbol{\mu}_k) = \mathbf{x}_n^\top \boldsymbol{\Lambda}_k \mathbf{x}_n - 2 \boldsymbol{\mu}_k^\top \boldsymbol{\Lambda}_k \mathbf{x}_n + \boldsymbol{\mu}_k^\top \boldsymbol{\Lambda}_k \boldsymbol{\mu}_k$$
Or is this the case when $\mathbf{x}_n$ comes from normal distribution, $\mathcal{N}(\mathbf{x}_n|\boldsymbol{\mu}_k,{\Lambda}_k)$ ?

I'm a bit confused because I know following equation is generally correct.
$$(\mathbf{x}_n - \boldsymbol{\alpha})^T (\mathbf{x}_n-\boldsymbol{\alpha}) = \mathbf{x}_n^T \mathbf{x}_n - \mathbf{x}_n^T \boldsymbol{\alpha} - \boldsymbol{\alpha}^T \mathbf{x}_n + \boldsymbol{\alpha}^T \boldsymbol{\alpha} $$

\subsection{Answer}
Your statement is true as long as the $\mathbf{\Lambda}_n$ matrix is symmetric. 

Expand the product and you'll get:
$$(\mathbf{x}_n -  \mathbf{\mu}_n)^T \mathbf{\Lambda}_n (\mathbf{x}_n -  \mathbf{\mu}_n) = \mathbf{x}_n ^T \mathbf{\Lambda}_n \mathbf{x}_n - \mathbf{x}_n^T \mathbf{\Lambda}_n \mathbf{\mu}_n -  \mathbf{\mu}_n^T \mathbf{\Lambda}_n \mathbf{x}_n + \mathbf{\mu}_n^T \mathbf{\Lambda}_n \mathbf{\mu}_n$$
The cross terms are equal only if the lambda matrix is symmetric. 

\end{document}
