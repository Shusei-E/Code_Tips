\documentclass[a4paper,10.5pt,dvipdfmx]{jarticle}  %jsarticleでも良いかも
%--脚注の設定
\usepackage{natbib}
\bibpunct[, ]{(}{)}{;}{and}{}{,} %本文での引用の体裁はここで整えられるみたい
\bibliographystyle{apsr2006-2}   %このagsmは編集済みなので注意。ジャーナルが太字にならないようにしてる。ブログ参照。
\usepackage{amsmath}
%--余白の設定   http://pcwide-jp.blogspot.co.uk/2009/07/latex.html  より
\setlength{\topmargin}{20mm}
\addtolength{\topmargin}{-1in}
\setlength{\oddsidemargin}{20mm}
\addtolength{\oddsidemargin}{-1in}
\setlength{\evensidemargin}{15mm}
\addtolength{\evensidemargin}{-1in}
\setlength{\textwidth}{170mm}
\setlength{\textheight}{254mm}
\setlength{\headsep}{0mm}
\setlength{\headheight}{0mm}
\setlength{\topskip}{0mm}
%--図の設定
%\usepackage[dvipdfmx]{graphicx} % PDFの利用もOKに
%\graphicspath{{./figures/}} %To add paths relative to the latexfile invoking the command
  %\usepackage[dvips]{graphicx}
%--行間の設定
\usepackage{setspace} 
\setstretch{1.13} % ページ全体の行間を設定
%コードの設定
\usepackage{listings}
\usepackage{color}
\definecolor{dkgreen}{rgb}{0,0.6,0}
\definecolor{mygray}{rgb}{0.5,0.5,0.5}
\definecolor{mauve}{rgb}{0.58,0,0.82}

\definecolor{codegreen}{rgb}{0,0.6,0}
\definecolor{codegray}{rgb}{0.5,0.5,0.5}
\definecolor{codepurple}{rgb}{0.58,0,0.82}
\definecolor{backgroundcolour}{rgb}{0.95,0.95,0.92}

\lstdefinestyle{Python} {
  language= Python, %ここを含めなければ、変に太字になったりしない
  aboveskip=2.5mm,
  belowskip=4.5mm,
  showstringspaces=false,
  columns=flexible,
  keepspaces=true,
  numbers=left,                    
  numbersep=5pt,    
  basicstyle={\small\ttfamily},
  commentstyle={\small\ttfamily},
  breaklines=true,
  breakatwhitespace=true
  tabsize=3
  backgroundcolor=\color{backgroundcolour},   
  commentstyle=\color{codegreen},
  keywordstyle=\color{magenta},
  numberstyle=\tiny\color{codegray},
  stringstyle=\color{codepurple},
	xleftmargin = 1.1cm,
  framexleftmargin = 1em
}
\lstdefinestyle{C} {
  language= C, %ここを含めなければ、変に太字になったりしない
  aboveskip=2.5mm,
  belowskip=4.5mm,
  showstringspaces=false,
  columns=flexible,
  keepspaces=true,
  numbers=left,                    
  numbersep=5pt,    
  basicstyle={\small\ttfamily},
  commentstyle={\small\ttfamily},
  breaklines=true,
  breakatwhitespace=true
  tabsize=3
  backgroundcolor=\color{backgroundcolour},   
  commentstyle=\color{codegreen},
  keywordstyle=\color{magenta},
  numberstyle=\tiny\color{codegray},
  stringstyle=\color{codepurple},
	xleftmargin = 1.1cm,
  framexleftmargin = 1em
}
\usepackage{xcolor}
\usepackage{framed}
\colorlet{shadecolor}{green!8}
%リンクの埋め込みを可能にする  \href{ **URL** }{表示テキスト}
\usepackage[dvipdfmx]{hyperref}
\usepackage{courier}
\usepackage{here}
%日本語環境用の設定を追加
\usepackage[english]{babel}
\renewcommand{\refname}{参考文献}
\renewcommand{\abstractname}{要約}
\usepackage{multirow}
\usepackage{placeins}
%数学用のフォント
\usepackage{amsmath} 
\usepackage{amssymb}
\usepackage{amsfonts} 
%tikz
\usepackage{tikz}
\usetikzlibrary{bayesnet}
%tightlist
\def\tightlist{
	\itemsep1pt
	\parskip1pt
	\parsep1pt
	\itemindent20pt
}
%subsubsubsection
\makeatletter
  \newcommand{\subsubsubsection}{\@startsection{paragraph}{4}{\z@}%
    {1.0\Cvs \@plus.5\Cdp \@minus.2\Cdp}%
    {.1\Cvs \@plus.3\Cdp}%
    {\reset@font\normalsize}
  }
  \makeatother
  \setcounter{secnumdepth}{4}
%-----------------
\begin{document}

\title{Variational Bayes Derivation of Latent Dirichlet Allocation}
\author{Simple LDA, not smoothed LDA}
\maketitle
%\begin{abstract}
%\end{abstract}

\section{Model}
\begin{figure}[H]
\centering
\begin{tikzpicture}

  % Nodes

  \node[obs]                   (w)      {$w$} ; %
  \node[latent, left=of w]    (z)      {$z$} ; %
  \node[latent, left=of z]    (theta)  {$\theta$}; %
  \node[latent, left=of theta] (alpha) {$\alpha$};
  \node[latent, above=of z] (beta) {$\beta$};

	% Edges
	\edge{alpha}{theta}; \edge{theta}{z}; \edge{theta}{z}; \edge{z}{w}; 
	\edge{beta}{w};

	% Plates
  \plate {plateN} { %
    (w)(z) %
  }{$N_d$}; %
	\plate{plateM}{
		(plateN)(theta)
	}{$M$};
\end{tikzpicture}
\caption{LDA Model}
\end{figure}
Variables:
\begin{itemize}
	\tightlist
	\item $M$: a number of document
	\item $N_d$: a number of words in document $d$
	\item $w_{d,i}$: a word
	\item $\beta$: (number of topic $K$) $\times$ (number of unique words $V$)
		\begin{itemize}
				\tightlist
			\item $\beta_{k,v} = p(w_{d,i} = v|z_{d,i}=k)$ is the probability of the word $v$ occuring given the topic $k$
			\item Caution: At least in the Python code, $\beta$ is $V \times K$ (maybe in C as well)
		\end{itemize}
	\item $z_{d,i}$: latent topic
\end{itemize}

\section{Derivation with Code}
\subsection{Evidence lower bound}
\begin{align}
	\log p(\boldsymbol{w}|\boldsymbol{\alpha}, \boldsymbol{\beta}) &= \log \int \sum_{\boldsymbol{z}} p(\boldsymbol{w}, \boldsymbol{z}, \boldsymbol{\theta} | \boldsymbol{\alpha}, \boldsymbol{\beta}) d \boldsymbol{\theta} \qquad ({\rm intdotuce \ latent\ variables})\\
&=\log \int \sum_{\boldsymbol w} q(\boldsymbol{z}, \boldsymbol{\theta}) \frac{p(\boldsymbol{w}, \boldsymbol{z}, \boldsymbol{\theta} | \boldsymbol{\alpha}, \boldsymbol{\beta})}{q(\boldsymbol{z}, \boldsymbol{\theta})} d\boldsymbol{\theta}\\
&\leq \int \sum_{\boldsymbol{z}} q(\boldsymbol{z}, \boldsymbol{\theta}) \log \frac{p(\boldsymbol{w}, \boldsymbol{z}, \boldsymbol{\theta} | \boldsymbol{\alpha}, \boldsymbol{\beta})}{q(\boldsymbol{z}, \boldsymbol{\theta})} d \boldsymbol{\theta} \qquad \because {\rm Jensen's\ Inequality}\\
&\equiv F[q(\boldsymbol{z}, \boldsymbol{\theta})]
\end{align}

From factorization assumption:
\begin{equation}
	q(\boldsymbol{z}, \boldsymbol{\theta}) = \left[ \prod_{d=1}^{M} \prod_{i=1}^{N_d} q(z_{d,i}) \right] \left[ \prod_{d=1}^{M} q(\boldsymbol{\theta}_d) \right]
\end{equation}

Expand the joint distribution using Bayes' Theorem:
\begin{align}
	p(\boldsymbol{w}, \boldsymbol{z}, \boldsymbol{\theta} | \boldsymbol{\alpha}, \boldsymbol{\beta}) &= p(\boldsymbol{w}| \boldsymbol{\alpha}, \boldsymbol{\beta}, \boldsymbol{z}, \boldsymbol{\theta}) \underbrace{p(\boldsymbol{z}, \boldsymbol{\theta} | \boldsymbol{\alpha}, \boldsymbol{\beta})}_{p(\boldsymbol{z}|\boldsymbol{\theta}, \boldsymbol{\alpha}, \boldsymbol{\beta}) p(\boldsymbol{\theta} | \boldsymbol{\alpha}, \boldsymbol{\beta})} \\
&= p(\boldsymbol{w} | \boldsymbol{z}, \boldsymbol{\beta}) p(\boldsymbol{z} | \boldsymbol{\theta}) p(\boldsymbol{\theta} | \boldsymbol{\alpha})  \qquad \because {\rm Graphical\ Model}\\
&= \left[ \prod_{d=1}^{M} \prod_{i=1}^{N_d} p(w_{d,i} | \beta_{z_{d,i}}) p(z_{d,i} | \boldsymbol{\theta}_d) \right] \left[ \prod_{d=1}^{M} p(\boldsymbol{\theta}_{d} | \boldsymbol{\alpha})  \right]
\end{align}

Evidence lower bound (ELBO) is:
\begin{align}
	F[q(\boldsymbol{z}, \boldsymbol{\theta})] &= \int \sum_{\boldsymbol{z}} q(\boldsymbol{z}) q(\boldsymbol{\theta}) \log \frac{p(\boldsymbol{w} | \boldsymbol{z}, \boldsymbol{\beta}) p(\boldsymbol{z} | \boldsymbol{\theta}) p(\boldsymbol{\theta} | \boldsymbol{\alpha})}{q(\boldsymbol{z}) q(\boldsymbol{\theta})} d \boldsymbol{\theta}\\
&= \int \sum_{\boldsymbol{z}} q(\boldsymbol{z}) q(\boldsymbol{\theta}) \log \frac{p(\boldsymbol{w} | \boldsymbol{z}, \boldsymbol{\beta}) p(\boldsymbol{z} | \boldsymbol{\theta})}{q(\boldsymbol{z})} d \boldsymbol{\theta} + \int \sum_{\boldsymbol{z}} q(\boldsymbol{z}) q(\boldsymbol{\theta}) \log \frac{p(\boldsymbol{\theta} | \boldsymbol{\alpha})}{q(\boldsymbol{\theta})} d \boldsymbol{\theta} \\
&=\int \sum_{\boldsymbol{z}} q(\boldsymbol{z}) q(\boldsymbol{\theta}) \log p(\boldsymbol{w} | \boldsymbol{z}, \boldsymbol{\beta}) p(\boldsymbol{z} | \boldsymbol{\theta}) d \boldsymbol{\theta} - \sum_{\boldsymbol{z}} q(\boldsymbol{z}) \log q (\boldsymbol{z}) + \int q(\boldsymbol{\theta}) \log \frac{p(\boldsymbol{\theta} | \boldsymbol{\alpha})}{q(\boldsymbol{\theta})} d \boldsymbol{\theta} \\
&\qquad\qquad\qquad\qquad\qquad\qquad\qquad\qquad\qquad\qquad\qquad ({\rm integrate\ out\ unrelated\ variables}) \nonumber \\
\begin{split}
&= \int \sum_{d=1}^{M} \sum_{i=1}^{N_d} q(z_{d,i}) q(\boldsymbol{\theta}_d) \log p(w_{d,i} | z_{d,i}, \beta) p(z_{d,i} | \boldsymbol{\theta}_d) d \boldsymbol{\theta}_d\\
&\qquad - \sum_{d=1}^{M} \sum_{i=1}^{N_d} \sum_{k=1}^{K} q(z_{d,i} = k) \log q(z_{d,i} = k)\\
&\qquad - \sum_{d=1}^{M} \underbrace{\int q(\boldsymbol{\theta}_d) \log \frac{q(\boldsymbol{\theta}_d)}{p(\boldsymbol{\theta}_d | \boldsymbol{\alpha})} d \boldsymbol{\theta}_d}_{{\rm KL}\left[ q(\boldsymbol{\theta}_d) || p(\boldsymbol{\theta}_d | \boldsymbol{\alpha}) \right]}
\end{split}
\end{align}

\subsection{Update Equation of $q(z_{d,i})$}
\subsubsection{Derivation}
\begin{align}
	\widetilde{F}[q(z_{d,i})] &= \sum_{k=1}^{K} q(z_{d,i} = k) \int q(\boldsymbol{\theta}_d) \log \{ p(w_{d,i} | z_{d,i}, \beta_{k,i}) p(z_{d,i} = k | \boldsymbol{\theta}_d) \} d \boldsymbol{\theta}_d - \sum_{k=1}^{K} q(z_{d,i}=k) \log q(z_{d,i}=k)
\end{align}
Use variational inference:
\begin{align}
	\frac{\delta \widetilde{F}[q(z_{d,i})]}{\delta q(z_{d,i} = k)} = \frac{\partial \widetilde{F}[q(z_{d,i})]}{\partial q(z_{d,i} = k)} &= \int q(\boldsymbol{\theta}_d) \log(\beta_{k, w_{d,i}}, \theta_{d,k}) d \boldsymbol{\theta}_d - \log q(z_{d,i} = k) - 1 = 0
\end{align}
Hence,
\begin{align}
	q(z_{d,i} = k) &\propto \exp \left[ \int q(\boldsymbol{\theta}_d) \log (\beta_{k, w_{d,i}} \theta_{d,k}) d \boldsymbol{\theta}_d \right] \\
&= \exp \left[ \int q (\boldsymbol{\theta}_d) \log(\beta_{k, w_{d,i}}) d \boldsymbol{\theta}_d \right] \exp \left[\int q(\boldsymbol{\theta}_d) \log(\theta_{d,k}) d \boldsymbol{\theta}_d  \right]\\
&= \beta_{k, w_{d,i}} \ \exp \left[ \mathbb{E}_{q(\boldsymbol{\theta}_d)} \log(\theta_{d,k}) \right]\\
&\propto \beta_{k, w_{d,i}} \ \frac{\exp \left[ \Psi( \xi_{d,k}^{\theta}) \right]}{\exp \left[ \Psi( \sum_{k^{`} = 1}^{K} \xi_{d,k^{'}}^{\theta}) \right]} \qquad \xi_{d,k}^{\theta} = \mathbb{E}_{q(\boldsymbol{z}_d)} [N_{d,k}] + \alpha_k \label{Eq:z_q}
\end{align}
Note $\Psi(\ )$ is a digamma function. If $p(\boldsymbol{\theta} | \boldsymbol{\alpha})$ is a $K$-dimensional Dirichlet distribution,$$ \mathbb{E}_{p(\boldsymbol{\theta} | \boldsymbol{\alpha})} [\log \theta_k] = \Psi(\alpha_k) - \Psi \left( \sum_{k=1}^{K} \alpha_k \right) $$

\subsubsection{Code}
\noindent
Caution: At least in the Python code, $\beta$ is $V \times K$ (maybe in C as well) \par
In Python,
\begin{lstlisting}[style=Python]
q = lda.mnormalize(matrix(beta[d[0],:]) * matrix(diag(exp(digamma(alpha0 + nt))[0])), 1)
\end{lstlisting}

In C,
\begin{lstlisting}[style=C]
/* In vbem.c */
/* vb-estep */
for (k = 0; k < K; k++)
	ap[k] = exp(psi(alpha[k] + nt[k]));
/* accumulate q */
for (l = 0; l < L; l++)
	for (k = 0; k < K; k++)
		q[l][k] = beta[d->id[l]][k] * ap[k];
/* normalize q */
for (l = 0; l < L; l++) {
	z = 0;
	for (k = 0; k < K; k++)
		z += q[l][k];
	for (k = 0; k < K; k++)
		q[l][k] /= z;
}
\end{lstlisting}
\texttt{ap[k]} is the second term in (\ref{Eq:z_q}).



\subsection{Update Equation of $q(\boldsymbol{\theta}_d)$}
\subsubsection{Derivation}
Again, ELBO is
\begin{align}
	F[q(\boldsymbol{z}, \boldsymbol{\theta})]  &= \int \sum_{d=1}^{M} \sum_{i=1}^{N_d} q(z_{d,i}) q(\boldsymbol{\theta}_d) \log p(w_{d,i} | z_{d,i}, \beta) p(z_{d,i} | \theta_{d}) d \boldsymbol{\theta}_d - \sum_{d=1}^{M} \sum_{i=1}^{N_d} \sum_{k=1}^{K} q(z_{d,i} = k) \log q(z_{d,i} = k) \\ &\qquad - \sum_{d=1}^{M} \underbrace{\int q(\boldsymbol{\theta}_d) \log \frac{q(\boldsymbol{\theta}_d)}{p(\boldsymbol{\theta}_d) | \boldsymbol{\alpha})} d \boldsymbol{\theta}_d}_{{\rm KL}\left[ q(\boldsymbol{\theta}_d) || p(\boldsymbol{\theta}_d | \boldsymbol{\alpha}) \right]} \nonumber
\end{align}
We use terms only related to $\boldsymbol{\theta}$.
\begin{align}
	\widetilde{F}[q(\boldsymbol{\theta})] = \int q(\boldsymbol{\theta}) \sum_{\boldsymbol{z}} q(\boldsymbol{z}) \log p(\boldsymbol{z} | \boldsymbol{\theta}) d \boldsymbol{\theta} - \int q(\boldsymbol{\theta}) \log \frac{q(\boldsymbol{\theta})}{p(\boldsymbol{\theta} | \boldsymbol{\alpha})} d \boldsymbol{\theta} \\
	\widetilde{F}[q(\boldsymbol{\theta}_d)] = \int q(\boldsymbol{\theta}_d) \sum_{\boldsymbol{z}} q(\boldsymbol{z}) \sum_{i=1}^{N_d} \log p(z_{d,i} | \boldsymbol{\theta}_d) d \boldsymbol{\theta}_d - \int q(\boldsymbol{\theta}_d) \log \frac{q(\boldsymbol{\theta}_d)}{p(\boldsymbol{\theta}_d | \boldsymbol{\alpha})} d \boldsymbol{\theta}_d
\end{align}
Using variational inference,
\begin{align}
	\frac{\delta \widetilde{F}[q(\boldsymbol{\theta}_d)]}{\delta q(\boldsymbol{\theta}_d)} = \frac{\partial \widetilde{F}[q(\boldsymbol{\theta}_d)]}{\partial q(\boldsymbol{\theta}_d)} = \sum_{\boldsymbol{z}} q(\boldsymbol{z}) \sum_{i=1}^{N_d} \log p(z_{d,i} | \boldsymbol{\theta}_d) - \log \frac{q(\boldsymbol{\theta}_d)}{p(\boldsymbol{\theta}_d | \boldsymbol{\alpha})} - 1 = 0
\end{align}
Before we move on, let's check some deformations:
\begin{itemize}
\item Dirichlet distribution
\begin{align}
	{\rm Dir}(\boldsymbol{\pi} | \boldsymbol{\alpha}) \equiv \frac{\Gamma(\sum_{k=1}^{K} \alpha_k)}{\prod_{k=1}^{K} \Gamma(\alpha_k)} \prod_{k=1}^{K} \pi_{k}^{\alpha_k - 1} 
\end{align}
\item If we consider a category $k$ in a document $d$, the average number of words that belong to the category $k$ under certain latent variables is
\begin{align}
	\mathbb{E}_{q(\boldsymbol{z}_d)}[N_{d,k}] = \sum_{i=1}^{N_d} q(z_{d,i} = k) \label{Eq:N_dk}
\end{align}
\item Remember $z_{d,i} \sim {\rm Multi}(\boldsymbol{\theta}_d)$ (Sato pp.26-27, Equation 2.1). Be careful that a word belongs to a category or not, so we can use $\delta(z_{d,i}=k)$ here.
\begin{align}
	p(z_{d,i} | \boldsymbol{\theta}_d) = \prod_{k=1}^{K} \theta_{d,k}^{\delta(z_{d,i} = k)}
\end{align}
\end{itemize}

Now, we can back to the variational inference
\begin{align}
	p(\boldsymbol{\theta}_d) &\propto p(\boldsymbol{\theta}_d | \boldsymbol{\alpha}) \exp \left[ \sum_{\boldsymbol{z}} q(\boldsymbol{z}) \sum_{i=1}^{N_d} \log p(z_{d,i} | \boldsymbol{\theta}_d) \right]\\
 &\propto \prod_{k=1}^{K} \theta_{d,k}^{\alpha_k -1} \exp \left[ \sum_{\boldsymbol{z}} q(\boldsymbol{z}) \sum_{i=1}^{N_d} \sum_{k=1}^{K} \delta({z_{d,i} = k}) \log \theta_{d,k} \right] \label{Eq:t_1} \\
&= \exp \left[ \sum_{k=1}^{K} (\alpha_k -1) \log \theta_{d,k} \right] \exp \left[ \sum_{k=1}^{K} \sum_{i=1}^{N_d} q(z_{d,i} = k) \log \theta_{d,k} \right] \label{Eq:t_2} \\
&= \exp \left[\sum_{k=1}^{K} (\alpha_k -1) \log \theta_{d,k}  \right] \exp \left[ \sum_{k=1}^{K} \mathbb{E}_{q(\boldsymbol{z}_d)}[N_{d,k}]  \log \theta_{d,k} \right] \\
&= \exp \left[ \sum_{k=1}^{K} (\mathbb{E}_{q(\boldsymbol{z}_d)}[N_{d,k}] + \alpha_k -1) \log \theta_{d,k} \right]\\
&= \prod_{k=1}^{K} \theta_{d,k}^{\mathbb{E}_{q(\boldsymbol{z}_d)}[N_{d,k}] + \alpha_k -1}
\end{align}
From (\ref{Eq:t_1}) to (\ref{Eq:t_2}), we marginalize the equation with respect to $q(\boldsymbol{z})$. For various $z_{d,i}$ in $\boldsymbol{z}$, some take $\delta({z_{d,i} = k})=0$ and other take $\delta({z_{d,i} = k})=1$. We only need to consider those that are $\delta({z_{d,i} = k})=1$. \par
If we define $\xi_{d,k}^{\theta} = \mathbb{E}_{q(\boldsymbol{z}_d)}[N_{d,k}] + \alpha_k$, $q(\boldsymbol{\theta}_d)$ is a Dirichlet distribution whose parameters are $\boldsymbol{\xi}_{d}^{\theta} = (\xi_{d,1}^{\theta}, \xi_{d,2}^{\theta}, \cdots, \xi_{d,K}^{\theta})$. We can easily normalize it:
\begin{align}
	q(\boldsymbol{\theta}_d | \boldsymbol{\xi}_{d}^{\theta}) = \frac{\Gamma (\sum_{k=1}^{K} \xi_{d,k}^{\theta})}{\prod_{k=1}^{K} \Gamma(\xi_{d,k}^{\theta})} \prod_{k=1}^{K} \theta_{d,k}^{\xi_{d,k}^{\theta} -1}
\end{align}


\subsubsection{Code}
\subsubsubsection{$\mathbb{E}_{q(\boldsymbol{z}_d)}[N_{d,k}]$ in Equation (\ref{Eq:N_dk})}
In the dataset, we only have how many times each word appears, so we calculate (number of times a word appears) $\times$ $q(z_{d,i}=k)$\par
In Python,
\begin{lstlisting}[style=Python]
nt = matrix(di[1]) * q
\end{lstlisting}

In C,
\begin{lstlisting}[style=C]
/* In vbem.c */
for (k = 0; k < K; k++) {
	z = 0;
	for (l = 0; l < L; l++)
		z += q[l][k] * d->cnt[l];
	nt[k] = z;
}
\end{lstlisting}

What are stored in \texttt{di[1]} and \texttt{d->cnt[1]} is the word count for each word in each document.

\section{More notes on code}
\subsection{\texttt{accum\_beta()}}
\noindent
This makes matrix $\beta$. Maybe it corresponds to \S A.4.1 in the original article. \par
In Python (mpre details are in \texttt{LDA-explanation.ipynb}),
\begin{lstlisting}[style=Python]
t = d[1]
betas[t[0],:] += matrix(diag(t[1])) * q 
\end{lstlisting}
Matrix \texttt{betas[t[0],:]} is (number of unique words in document no1) $\times$  (number of class (category). \texttt{t[0]} is word ID.

In C,
\begin{lstlisting}[style=C]
/* in learn.c */
int i, k;
int n = dp->len;

for (i = 0; i < n; i++)
	for (k = 0; k < K; k++)
		betas[dp->id[i]][k] += q[i][k] * dp->cnt[i];
\end{lstlisting}


\section*{Reference}
\begin{enumerate}
	\item Blei et al., "Latent Dirichlet Allocation", The Journal of Machine Learning Research, 2003.
	\item Mochihashi, Daichi. "lda, a Latent Dirichlet Allocation package" at http://chasen.org/~daiti-m/dist/lda/ for C code
	\item Sato, Makoto. 「PythonでLDAを実装してみる」 at http://satomacoto.blogspot.jp/2009/12/pythonlda.html for Python code
\end{enumerate}
\end{document}
